\documentclass[11pt]{article}
\usepackage{amsmath}
\usepackage{amssymb}
\usepackage{mathpartir}
\usepackage{tabularx}

% To produce a letter size output. Otherwise will be A4 size.
\usepackage[letterpaper]{geometry}

% For enumerated lists using letters: a. b. etc.
\usepackage{enumitem}

\topmargin -.5in
\textheight 9in
\oddsidemargin -.25in
\evensidemargin -.25in
\textwidth 7in

\begin{document}

% Edit the following putting your first and last names and replace XXX with your lab section (e.g., F3L).
\author{Joshua Chin\\
Lab 021-CSE 015 01L}

% Edit the following replacing X with the HW number.
\title{CSE 15: Discrete Mathematics\\
Fall 2021\\
Homework \#6\\
Solution}

% Put today's date in the following.
\date{November 28, 2021}
\maketitle

% ========== Begin questions here
\begin{enumerate}

\item
\textbf{Question 1:} Recursively defined functions \\
Find $f(1)-f(5)$ if $f(n)$ is defined recursively as $f(0) = 3$
\begin{enumerate}[label=(\alph*)]
    \item
    $f(n+1) = -2f(n)$
    \begin{itemize}
        \item 
        $f(1)= -2(3) = -6$
        \item 
        $f(2) = -2(-6) = 12$
        \item 
        $f(3) = -2(12) = -24$
        \item 
        $f(4) = -2(-24) = 48$
        \item 
        $f(5) = -2(48) = -96$
    \end{itemize}
    \item 
    $f(n+1) = 3f(n)+7$
    \begin{itemize}
        \item 
        $f(1) = 3(3)+7 = 16$
        \item
        $f(2) = 3(16)+7 = 55$
        \item 
        $f(3) = 3(55)+7 = 172$
        \item
        $f(4) = 3(172)+7 = 523$
        \item 
        $f(5) = 3(523)+7 = 1576$
    \end{itemize}
    \item 
    $f(n+1) = f(n)^2-2f(n)-2$
    \begin{itemize}
        \item 
        $f(1) = 3^2-2(3)-2 = 1$
        \item 
        $f(2) = 1^2-2(1)-2 = -3$
        \item 
        $f(3) = -3^2-2(-3)-2 = 13$
        \item 
        $f(4) = -5^2-2(-3)-2 = 141$
        \item 
        $f(5) = -17^2-2(-17)-2 = 19597$
    \end{itemize}
    \item 
    $f(n+1)=3^\frac{f(n)}{3}$
    \begin{itemize}
        \item 
        $f(1) = 3^\frac{3}{3} = 3$
        \item 
        $f(2) = 3^\frac{3}{3} = 3$
        \item 
        $f(3) = 3^\frac{3}{3} = 3$
        \item 
        $f(4) = 3^\frac{3}{3} = 3$
        \item 
        $f(5) = 3^\frac{3}{3} = 3$
    \end{itemize}
\end{enumerate}
\item
\textbf{Question 2:} Recursively defined sequences
\begin{enumerate}[label=(\alph*)]
    \item
    $a_n = 4n-2$
    \begin{itemize}
        \item 
        Base case: $a_1 = 4(1)-2 = 2$ \\
        Next 3: $a_2 = 6, a_3 = 10, a_4 = 14$
        \item 
        Recursive formula: $a_n = a_{n-1} + 4$
    \end{itemize}
    \item 
    $a_n = 1+ (-1)^n$
    \begin{itemize}
        \item 
        Base case: $a_1 = 1 + (-1)^1 = 0$ \\
        Next 3: $a_2 = 2, a_3 = 0, a_4 = 2$
        \item 
        Recursive formula: $a_n =2-a_{n-1}$
    \end{itemize}
    \item 
    $a_n = n(n-1)$
    \begin{itemize}
        \item 
        Base case: $a_1 = 1(1-1) = 0$\\
        Next 3: $a_2 = 2, a_3 = 6, a_4 = 12$
        \item 
        Recursive formula: $a_n = a_{n-1} + 2n-2$
    \end{itemize}
    \item 
    $a_n = n^2$ 
    \begin{itemize}
        \item 
        Base case: $a_1 = 1^2 = 1$ \\
        Next 3: $a_2 = 4, a_3 = 9, a_4 = 16$
        \item 
        Recursive formula: $a_n = a_{n-1} + (a_{n-1} - a_{n-2} +2)$
    \end{itemize}
\end{enumerate}
\item
\textbf{Question 3:} Recursively defined sets\\
S is a set of strings made up of an equivalent amount of 0 and 1, this much is true. \\
However, when aknowledging the inductive step, 0x1 is now restricting how the 0 and 1s can be ordered.x is simply part of the string, which means x is capable of being 0 or 1 respectively. \\
the difference is, it must be in between 0 and 1, and neither 0 and 1 can be swapped with eachother.
Examples of this: 001 or 011.

\end{enumerate}
\end{document}
