\documentclass[11pt]{article}
\usepackage{amsmath}
\usepackage{amssymb}
\usepackage{mathpartir}
\usepackage{tabularx}

% To produce a letter size output. Otherwise will be A4 size.
\usepackage[letterpaper]{geometry}

% For enumerated lists using letters: a. b. etc.
\usepackage{enumitem}

\topmargin -.5in
\textheight 9in
\oddsidemargin -.25in
\evensidemargin -.25in
\textwidth 7in

\begin{document}

% Edit the following putting your first and last names and replace XXX with your lab section (e.g., F3L).
\author{Joshua Chin\\
Lab 021-CSE 015 01L}

% Edit the following replacing X with the HW number.
\title{CSE 15: Discrete Mathematics\\
Fall 2021\\
Homework \#7\\
Solution}

% Put today's date in the following.
\date{December 10, 2021}
\maketitle

% ========== Begin questions here
\begin{enumerate}

\item
\textbf{Question 1:} Asymptotic Notation \\
Inorder to determine complexity, you ignore all constants and keep track of the variable $n$. So if anywhere in the function there exists any $n^2$, the complexity is $O(n^2)$
\begin{enumerate}[label=(\alph*)]
    \item
    $f(n) = 178n+45$ \\
    Based on the rule above, in this case there is only one power of $n$. Therefore, it is not $O(n^2)$. 
    \item 
    $f(n) = nlogn+12$ \\
    Based on the rule above, in this case there is a single power of $n$ and a logarithm of $n$. Therefore, it is not $O(n^2)$.
    \item 
    $f(n) = 34n^2+34n+34$ \\
    Based on the rule above, in this case we see $n$ with a power of 2. Therefore, this function has a complexity of $O(n^2)$.
    \item 
    $f(n) = \sqrt{n}+2 $ \\
    Based on the rule above, in this case there is only a square root of $n$. Therefore, it is not $O(n^2)$.
    \item 
    $f(n) = 0.001n^3+72n$ \\
    Based on the rule above, in this case we see $n$ with a power of 3 and $n$ with a power of n. Therefore it is not $O(n^2)$.
\end{enumerate}  

\item
\textbf{Question 2:} Asymptotic Notation 2 \\
This is the order of the 9 functions in from fastest Big-O complexity to slowest: 
\begin{enumerate}[label=(\alph*)]
    \item
    $logn$
    \item
    $\sqrt{n}$
    \item 
    $n$
    \item 
    $nlogn$
    \item 
    $n^2$
    \item 
    $n^2logn$
    \item 
    $n^4$
    \item
    $2^n$
    \item 
    $3^n$
\end{enumerate}
\item
\textbf{Question 3:} Asymptotic Growth  \\
For this, we find the highest value of n that Computer A and B can solve within an hour with the the inequality $f(n) \leq g(n)$ where $f(n)$ is the algorithm and $g(n)$ is the computer.\\
Since we are inluding all numbers up to and including the highest value, simply equating both sides of the inequality will also work.
\begin{enumerate}[label=(\alph*)]
    \item
    Computer A: $3.6 * 10^9$
    \begin{enumerate}[label=(\alph*)]
        \item 
        Algorithm 1: $5n^2+34n+12$ \\
        $n = 26829$, it is given.
        \item 
        Algorithm 2: $10n+4$ \\
        $10n + 4 = 3.9 * 10^9$\\
        $10n = 3600000000 - 4$ \\
        $n = 3599999996/10$ \\
        $n = 359999999.6$ 
        \item 
        Algorithm 2: $2^n$\\
        $2^n = 3.9 * 10^9$\\
        $nln(2) = ln(3600000000)$\\
        $n = \frac{ln(3600000000)}{ln(2)}$\\
        $n \approx  31.75$
    \end{enumerate}
    \item 
    Computer B: $3.6 * 10^{11}$
    \begin{enumerate}[label=(\alph*)]
        \item 
        Algorithm 1: $5n^2+34n+12$ \\
        $5n^2+34n+12 = 3.6*10^{11}$ \\
        $5n^2+34+360000000012 = 0$ \\
        Solve with quadratic formula: $n \approx 268234.76$
        \item 
        Algorithm 2: $10n + 4$ \\
        $10n+4 = 3.6*10^{11}$\\
        $10n = 359999999996$\\
        $n = 359999999996/10$ \\
        $n = 35999999999.6$
        \item 
        Algorithm 2: $2^n$\\
        $2^n = 3.6 * 10^{11}$\\
        $nln(2) = ln(360000000000)$\\
        $n = \frac{ln(360000000000)}{ln(2)}$\\
        $n \approx 38.39$
    \end{enumerate}
\end{enumerate}
\end{enumerate}
\end{document}
