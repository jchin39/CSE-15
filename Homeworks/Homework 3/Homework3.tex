\documentclass[11pt]{article}
\usepackage{amsmath}
\usepackage{amssymb}
\usepackage{mathpartir}
\usepackage{tabularx}

% To produce a letter size output. Otherwise will be A4 size.
\usepackage[letterpaper]{geometry}

% For enumerated lists using letters: a. b. etc.
\usepackage{enumitem}

\topmargin -.5in
\textheight 9in
\oddsidemargin -.25in
\evensidemargin -.25in
\textwidth 7in

\begin{document}

% Edit the following putting your first and last names and replace XXX with your lab section (e.g., F3L).
\author{Joshua Chin\\
Lab 021-CSE 015 01L}

% Edit the following replacing X with the HW number.
\title{CSE 15: Discrete Mathematics\\
Fall 2021\\
Homework \#3\\
Solution}

% Put today's date in the following.
\date{October 15, 2021}
\maketitle

% ========== Begin questions here
\begin{enumerate}

\item
\textbf{Question 1:} Rules of Inference \\
Convert: 
\begin{equation*}
\mprset{flushleft}
\inferrule
    {\textrm{If Jane does not fly, then she is not a bird.}\\\\ \textrm{Jane is a bird.}}
    {\therefore \textrm{Jane flies.}}
\end{equation*}
Into argument form: 
\begin{equation*}
\mprset{flushleft}
\inferrule
    {\lnot q \rightarrow \lnot p \\\\ p}
    {\therefore r}
\end{equation*}

\item 
\textbf{Question 2:} More Rules of Inference \\
For each of the following arguments, write w hich rule of inference is used.
\begin{enumerate}[label=(\alph*)]
\item 
Bats can fly($p$) and are mammals($q$). Therefore bats are mammals($q$). \\
$
\mprset{flushleft}
\inferrule
    {p \wedge q}
    {\therefore q}
$\\
This is Simplification.
\item 
Pigs are mammals($q$) or birds($p$). Pigs are not birds($\lnot p$). Therefore pigs are mammals($q$).\\ 
$
\mprset{flushleft}
\inferrule
    {p \vee q \\\\ \lnot p}
    {\therefore q}
$\\
This is Disjunctive Syllogism.
\item 
Jack is a cse major($p$). Jack is a freshmen($q$). Therefore Jack is a CSE major and a freshmen($p \wedge q$).\\
$
\mprset{flushleft}
\inferrule
    {p \\\\ q}
    {\therefore p \wedge q}
$\\
This is Conjunction.
\item
Mary is a CSE major($p$). Therefore Mary is a CSE major or Mary is a History major($p \vee q$).\\
$
\mprset{flushleft}
\inferrule
    {p}
    {\therefore p \vee q}
$\\
This is Addition.
\item 
If I go hiking, I will sweat a lot($p \rightarrow q$). If I sweat a lot, I will lose weight($q \rightarrow r$). Therefore, if I go hiking, I will lose weight($p \rightarrow r$).\\
$
\mprset{flushleft}
\inferrule
    {p \rightarrow q \\\\ q \rightarrow r}
    {\therefore p \rightarrow r}
$\\
This is Hypothetical Syllogism.
\end{enumerate}
\item
\textbf{Question 3:} Checking arguments. \\
State whether the following arguments are correct or not with proof.
\begin{enumerate}[label=(\alph*)]
\item 
If it is sunny, then I will go swimming. It is not sunny. Therefore  I will not go swimming.\\
Argument form: \\
$
\mprset{flushleft}
\inferrule
    {p \rightarrow q \\\\ \lnot p}
    {\therefore \lnot q}
$\\
This argument is valid. p needs to be true in order for q to be true. Since p is not true, q cannot be true.
\item 
If it is Sunday, then I will go to the park. It is not sunday. Therefore I will not go to the park.\\
$
\mprset{flushleft}
\inferrule
    {p \rightarrow q \\\\ \lnot q}
    {\therefore \lnot p}
$\\
This argument is invalid. p needs to be true only for q to be true. While q is not true, that does not mean p cannot be true.
\item 
I will pass the class if and only if I score at least 6F percent on the final exam. I scored 55 percent on the final exam. Therefore, I will not pass the class.\\
$
\mprset{flushleft}
\inferrule
    {p \leftrightarrow q \\\\ \lnot q}
    {\therefore \lnot p}
$\\
This argument is valid. Both p and q need eachother to be true. Since one is not true, the other cannot be true.
\end{enumerate}
\item 

\textbf{Question 4:} Proof by Contraposition. \\
if $n$ is an integer and $n^2$ is odd, then $n$ is odd.\\
This is: 
\begin{equation*}
    p \rightarrow q
\end{equation*}
The contrapositive of this is:
\begin{equation*}
    \lnot p \rightarrow \lnot q
\end{equation*}
This is the same thing as: if $n$ is an integer and $n^2$ is even, then $n$ is even. This is a true statement.

\item 
\textbf{Question 5:} Proof by Cases \\
Prove: $((pT \land p2 \land p3) \to q) <-> ((pT \to q) \land (p2 \to q) \land (p3 \to q))$ \\
\begin{tabular}{|c|c|c|c|c|c|c|c|}
    \hline
    $pT$ & $p2$ & $p3$ & $q$ & $pT \to q$ & $p2 \to q$ & $p3 \to q$ & $pT \land p2 \land p3 \to q$ \\
    \hline
    F & F & F & F & T & T & T & F\\
    \hline
    F & F & F & T & T & T & T & F\\
    \hline
    F & F & T & F & T & T & F & F\\
    \hline
    F & F & T & T & T & T & T & F\\
    \hline
    F & T & F & F & T & F & T & F\\
    \hline
    F & T & F & T & T & T & T & F\\
    \hline
    F & T & T & F & T & F & F & F\\
    \hline
    F & T & T & T & T & T & T & F\\
    \hline
    T & F & F & F & F & T & T & F\\
    \hline
    T & F & F & T & T & T & T & F\\
    \hline
    T & F & T & F & F & T & F & F\\
    \hline
    T & F & T & T & T & T & T & F\\
    \hline
    T & T & F & F & F & F & T & T\\
    \hline
    T & T & F & T & T & T & T & T\\
    \hline
    T & T & T & F & F & F & F & F\\
    \hline
    T & T & T & T & T & T & T & T\\
    \hline
\end{tabular} \\
\begin{tabular}{|c|c|}
    \hline
    $(pT \to q) \land (p2 \to q) \land (p3 \to q)$ & $((pT \land p2 \land p3) \to q) <-> ((pT \to q) \land (p2 \to q) \land (p3 \to q))$ \\
    \hline
    T & T \\
    \hline
    T & T \\
    \hline
    T&T \\
    \hline
    F&T \\
    \hline
    F&T \\
    \hline
    F&T \\
    \hline
    F&T \\
    \hline
    F&T \\
    \hline
    F&T \\
    \hline
    T&T \\
    \hline
\end{tabular}\\
Judging by the final column of the truth table, this is a tautology.
\end{enumerate}
\end{document}
