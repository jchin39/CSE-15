\documentclass[11pt]{article}
\usepackage{amsmath}
\usepackage{amssymb}
\usepackage{mathpartir}
\usepackage{tabularx}

% To produce a letter size output. Otherwise will be A4 size.
\usepackage[letterpaper]{geometry}

% For enumerated lists using letters: a. b. etc.
\usepackage{enumitem}

\topmargin -.5in
\textheight 9in
\oddsidemargin -.25in
\evensidemargin -.25in
\textwidth 7in

\begin{document}

% Edit the following putting your first and last names and replace XXX with your lab section (e.g., F3L).
\author{Joshua Chin\\
Lab 021-CSE 015 01L}

% Edit the following replacing X with the HW number.
\title{CSE 15: Discrete Mathematics\\
Fall 2021\\
Homework \#5\\
Solution}

% Put today's date in the following.
\date{November 14, 2021}
\maketitle

% ========== Begin questions here
\begin{enumerate}

\item
\textbf{Question 1:} Mathematical Induction 1 \\
Let $P(n)$ be the statement that
\begin{equation*}
1^3 + 2^3 + 3^3 + \dots + n^3 = \left( \frac{n(n+1)}{2} \right)^2
\end{equation*}
Answer the following questions: 
\begin{enumerate}[label=(\alph*)]
    \item
    What is the statement $P(1)$? \\
    $P(1) = \left( \frac{1(1+1)}{2} \right)^2$
    \item 
    Basis of induction: show that the statement $P(1)$ is true. \\
    $P(1) = \left( \frac{1(1+1)}{2} \right)^2$ \\
    $P(1) = \left( \frac{2}{2} \right)^2$ \\
    $P(1) = \left( 1 \right)^2$ \\
    $P(1) = 1$\\
    $1^3 = 1$
    \item
    Inductive hypothesis for $P(n)$ \\
    $P(k) = \left( \frac{k(k+1)}{2} \right)^2$ 
    \item
    The inductuve step for $P(n)$ \\
    $P(k+1) = \left( \frac{k+1(k+2)}{2} \right)^2$ \\
    $P(1+1) = \left( \frac{1+1(1+2)}{2} \right)^2$ \\
    $P(2) = \left( \frac{2(3)}{2} \right)^2$ \\
    $P(2) = \left( \frac{6}{2} \right)^2$ \\
    $P(2) = \left( 3 \right)^2$ \\
    $P(2) = 9$ \\
    $1^3 + 2^3 = 9$
\end{enumerate}
\item
\textbf{Question 2:} Mathematical Induction 2 \\
Compute the sum of the first even natural numbers and create a formula for their sum.
\begin{enumerate}[label=(\alph*)]
    \item
    First even number is 0: $P(1) = 0$ \\
    Second even number is 2: $P(2) = 0 + 2 = 2$\\
    Third even number is 4: $P(3) = 0 + 2 + 4 = 6$\\
    Fourth even number is 6: $P(4) = 0 + 2 + 4 + 6 = 12$\\
    Formula: 
\begin{equation*}
    P(n) = n(n-1)
\end{equation*}
    \item 
    $P(k) = k(k-1) \rightarrow P(k+1) = k+1(k+1-1)$\\
    Base case: $P(1) = 1(1-1) = 0$ \\
    Induction step 1: $P(1+1) = 1+1(1+1-1) = 2$ \\
    $P(2) = 2(1) = 2$ \\
    Induction step 2: $P(2+1) = 2+1(2+1-1) = 6$ \\
    $P(3) = 3(2) = 6$ \\
    Induction step 3: $P(3+1) = 3+1(3+1-1) = 12$ \\
    $P(4) = 4(3) = 12$ 
\end{enumerate}
\item
\textbf{Question 3:} Mathematical Induction 3 \\
Use Mathematical Induction to prove this inequality: 
\begin{equation*}
    n! < n^n
\end{equation*}
\begin{enumerate}[label=(\alph*)]
    \item
    What is the statement $P(2)$? \\
    $2! < 2^2$
    \item 
    Proof for P(2): \\
    $2! = 1 * 2$\\
    $2^2 = 4$\\
    $2 < 4$ 
    \item 
    Inductive hypothesis: \\
    $P(k) = k! < k^k$ where $k > 1$ \\
    $P(k+1) = (k+1)! < (k+1)^{k+1}$ where $k > 1$ \\
    $P(2) = 2! < 2^2 = 2 < 4$ \\
    $P(2+1) = (2+1)! < (2+1)^{2+1}$ \\
    $P(3) = 3! < 3^3 = 6 < 27$
\end{enumerate}
\end{enumerate}
\end{document}
