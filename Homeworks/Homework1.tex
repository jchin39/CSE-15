\documentclass[11pt]{article}

% To produce a letter size output. Otherwise will be A4 size.
\usepackage[letterpaper]{geometry}

% For enumerated lists using letters: a. b. etc.
\usepackage{enumitem}

\topmargin -.5in
\textheight 9in
\oddsidemargin -.25in
\evensidemargin -.25in
\textwidth 7in

\begin{document}

% Edit the following putting your first and last names and replace XXX with your lab section (e.g., 03L).
\author{Joshua Chin\\
Lab F21-CSE 015 07L}

% Edit the following replacing X with the HW number.
\title{CSE 015: Discrete Mathematics\\
Fall 2021\\
Homework \#1\\
Solution}

% Put today's date in the following.
\date{September 24, 2021}
\maketitle

% ========== Begin questions here
\begin{enumerate}

\item
\textbf{Question 1:}
Translating English Sentences \\
Translate the following compound propositions to english. \\
$p$: "Josh scored 100\% in the CSE015 Final" \\
$q$: "Josh scored at least 90\% in the labs."\\
$r$: "Josh receives an A+ in CSE015." \\
$t$: "Josh is a CSE major.\\

\begin{enumerate}[label=(\alph*)]
\item
$\lnot t$ = "It is not the case that Josh is a CSE Major
\item
$\lnot q$ = "It is not the case that Josh scored at least 90\% in the labs
\item
$(p \vee q) \rightarrow r$: "If Josh scored 100\% in the CSE015 final or if Josh scored 90\% in the labs, then Josh gets an A+
\item 
$(p\wedge q) \rightarrow r$: "If Josh scored 100\% in the CSE015 final and if Josh scored at least 90\% in the labs, then Josh receives an A+ in CSE015
\item 
$\lnot(t \rightarrow r)$: "It is not the case that if Josh is a CSE major then he will receive an A+ in CSE015.
\end{enumerate}


\item
\textbf{Question 2:}
Truth Tables
\begin{enumerate}[label=(\alph*)]
\item
Truth table for: $p \oplus (q \vee \lnot r)$ \\
\begin{tabular}{|c|c|c|c|c|c|} \hline
$p$ & $q$ & $r$ & $\lnot r$ & $q \vee \lnot r$ & $p \oplus (q \vee \lnot r)$ \\
\hline
T & T & T & F & T & F \\
\hline
T & T & F & T & T & F \\
\hline
T & F & T & F & F & T \\
\hline
T & F & F & T & T & F \\
\hline
F & T & T & F & T & T \\
\hline
F & T & F & T & T & T \\
\hline
F & F & T & F & F & F \\
\hline
F & F & F & T & T & T \\
\hline
\end{tabular}
\item
Truth table for: $(p \vee q) \rightarrow (\lnot r \vee p)$ \\
\begin{tabular}{|c|c|c|c|c|c|c|} \hline
$p$ & $q$ & $r$ & $p \vee q$ & $\lnot r$ & $\lnot r \vee p$ & $(p \vee q) \rightarrow (\lnot r \vee p)$ \\
\hline
T & T & T & T & F & T & T \\
\hline
T & T & F & T & T & T & T \\
\hline
T & F & T & T & T & T & T \\
\hline
T & F & F & T & T & T & T \\
\hline
F & T & T & T & F & F & F \\
\hline
F & T & F & T & T & T & T \\
\hline
F & F & T & F & F & F & T \\
\hline
F & F & F & F & T & T & T \\
\hline
\end{tabular}

\item 
Truth table for: $((p \rightarrow q) \wedge p) \rightarrow q$ \\
\begin{tabular}{|c|c|c|c|c|} \hline
$p$ & $q$ & $p \rightarrow q$ & $(p \rightarrow q) \wedge p$ & $((p \rightarrow q) \wedge p) \rightarrow q$ \\
\hline
T & T & T & T & T \\
\hline
T & F & F & F & T \\
\hline
F & T & T & F & T \\
\hline
F & F & T & F & T \\
\hline
\end{tabular}
\end{enumerate}

\item
\textbf{Question 3:}
Logical Equivalencies

\begin{enumerate}[label=(\alph*)]
\item
Truth table proof for: $p \vee (q \wedge r) \equiv (p \vee q) \wedge (p \vee r)$ (Distributive Property) \\
\begin{tabular}{|c|c|c|c|c|c|c|c|c|} \hline
$p$ & $q$ & $r$ & $q \wedge r$ & $p \vee (q \wedge r)$ & $p \vee q$ & $p \vee r$ & $(p \vee q) \wedge (p \vee r)$ & $p \vee (q \wedge r) \equiv (p \vee q) \wedge (p \vee r)$ \\
\hline
T&T&T&T&T&T&T&T&T \\ %When I found that you didn't need spaces...sheesh
\hline
T&T&F&F&T&T&T&T&T \\
\hline
T&F&T&F&T&T&T&T&T \\ 
\hline
T&F&F&F&T&T&T&T&T \\
\hline
F&T&T&T&T&T&T&T&T \\
\hline
F&T&F&F&F&T&F&F&T \\
\hline
F&F&T&F&F&F&T&F&T \\
\hline
F&F&F&F&F&F&F&F&T \\
\hline
\end{tabular}
\item 
Truth table proof for: $(p \rightarrow q) \wedge (p \rightarrow r) \equiv (q \wedge r)$ \\
\begin{tabular}{|c|c|c|c|c|c|c|c|} \hline
$p$ & $q$ & $r$ & $p \rightarrow q$ & $p \rightarrow r$ & $(p \rightarrow q) \wedge (p \rightarrow r)$ & $q \wedge r$ & $(p \rightarrow q) \wedge (p \rightarrow r) \equiv (q \wedge r)$ \\
\hline
T&T&T&T&T&T&T&T \\
\hline
T&T&F&T&F&F&F&T \\
\hline
T&F&T&F&T&F&F&T \\
\hline
T&F&F&F&F&F&F&T \\
\hline
F&T&T&T&T&T&T&T \\
\hline
F&T&F&T&T&T&F&F \\
\hline
F&F&T&T&T&T&F&F \\
\hline
F&F&F&T&T&T&F&F \\
\hline
\end{tabular}
\end{enumerate}

\item
\textbf{Question 4:}
Tautologies, Contingencies, and Contradictions\\

\begin{enumerate}[label=(\alph*)]
\item
$p \rightarrow (p \vee q)$: Is a tautology because the compound proposition is always true. \\
Proof: \\
\begin{tabular}{|c|c|c|c|} \hline
$p$ & $p \wedge q$ & $\lnot p$ & $(p \wedge q) \rightarrow \lnot p$ \\
\hline
T&T&T&T \\
\hline
T&F&T&T \\
\hline
F&T&T&T \\ 
\hline
F&F&F&T \\
\hline
\end{tabular}
\item 
$(p \wedge q) \rightarrow \lnot p$: Contingency because the compound proposition is not always true or false. \\
Proof: \\
\begin{tabular}{|c|c|c|c|c|} \hline
$p$ & $q$ & $p \wedge q$ & $\lnot p$ & $(p \wedge q) \rightarrow \lnot p$ \\
\hline
T&T&T&F&F \\
\hline
T&F&F&F&T \\
\hline
F&T&F&T&T \\
\hline
F&F&F&T&T \\
\hline
\end{tabular}
\item 
$(p \rightarrow (q \vee r)) \rightarrow (\lnot q \vee p)$: Contingency because the compound proposition is not always true or false. \\
Proof: \\
\begin{tabular}{|c|c|c|c|c|c|c|c|} \hline
$p$ & $q$ & $r$ & $q \vee r$ & $p \rightarrow (q \vee r)$ & $ \lnot q$ & $\lnot q \vee p$ & $(p \rightarrow (q\vee r)) \rightarrow (\lnot q \vee p)$ \\
\hline
T&T&T&T&T&F&T&T \\
\hline
T&T&F&T&T&F&T&T \\ 
\hline
T&F&T&T&T&T&T&T \\
\hline
T&F&F&F&F&T&T&T \\
\hline
F&T&T&T&T&F&F&F \\
\hline
F&T&F&T&T&F&F&F \\ 
\hline
F&F&T&T&T&T&T&T \\
\hline
F&F&F&F&T&T&T&T \\
\hline
\end{tabular}
\end{enumerate}
\item
\textbf{Question 5:}De Morgan's Laws\\
Using De Morgan's laws, rewrite the following sentences in English.\\
\begin{equation}
 \lnot(p \wedge q) \equiv (\lnot p \vee \lnot q)
\end{equation}
\begin{equation}
\lnot(p \vee q) \equiv (\lnot p \wedge \lnot q) 
\end{equation}
\begin{enumerate}[label=(\alph*)]
\item
You cannot be late and you cannot smoke. \\
\begin{itemize}

\item 
$\lnot p$: "It is not the case that you can be late.
\item 
$\lnot q$: "It is not the case that you can smoke.
\item 
$p$: You can be late.
\item 
$q$: You can smoke.
\end{itemize}
By De Morgan's law, this sentence can be translated to: "It is not the case that you can be late or smoke" \\
\item 
It is not the case that you can take an annuity and you can take a lump sum. \\
\begin{itemize}
\item
$\lnot p$: "You cannot take an annuity"
\item 
$\lnot q$: "You cannot take a lump sum"
\item 
$p$: "You can take an annuity"
\item 
$q$: "You can take a lump sum
\end{itemize}
By De Morgan's law, this sentence can be translated to: You cannot take an annuity or a lump sum"
\end{enumerate}


\end{enumerate}

\end{document}
