\documentclass[11pt]{article}

% To produce a letter size output. Otherwise will be A4 size.
\usepackage[letterpaper]{geometry}

% For enumerated lists using letters: a. b. etc.
\usepackage{enumitem}

\topmargin -.5in
\textheight 9in
\oddsidemargin -.25in
\evensidemargin -.25in
\textwidth 7in

\begin{document}

% Edit the following putting your first and last names and replace XXX with your lab section (e.g., 03L).
\author{Joshua Chin\\
Lab F21-CSE 015 07L}

% Edit the following replacing X with the HW number.
\title{CSE 015: Discrete Mathematics\\
Fall 2021\\
Homework \#2\\
Solution}

% Put today's date in the following.
\date{October 1, 2021}
\maketitle

% ========== Begin questions here
\begin{enumerate}

\item
\textbf{Question 1:} Quantifiers \\
Let $P(x)$ = $x < x^3 $. \\
Determine truth values for each of the following formulas: 
\begin{enumerate}[label=(\alph*)]
\item 
$P(2)$ 
\begin{itemize}
    \item 
    True. $2 < 2^3 \rightarrow 2 < 8$
\end{itemize} 
\item 
$P(-1)$
\begin{itemize}
    \item 
    False. $-1 < -1^3 \rightarrow -1 = -1$
\end{itemize}  
\item 
$\forall xP(x)$ 
\begin{itemize}
    \item 
    False. This is false whenever $x \leq  -1, x = 0, x = 1 \rightarrow -1 = -1^3,  0 = 0^3,  1 = 1^3$
\end{itemize} 
\item 
$\exists xP(x)$ 
\begin{itemize}
    \item 
    True. We already know this is true, as part a exhibits a value of x where P(x) is valid.
\end{itemize} 
\item 
$\exists !xP(x)$ 
\begin{itemize}
    \item 
    False. Since we know there is at least 1 valid x, in part a, and that pard d is true, this is false.
\end{itemize} 
\end{enumerate}

\item 
\textbf{Question 2:} Translating English Sentences into Formulas \\
Consider the following predicates: \\
$S(x)$: $x$ is a student in CSE015. \\
$M(x)$: $x$ plays a musical instrument. \\
Let domain be the set of all people. Using these predicates, translate the following sentences into formulas using hte appropriate quantifiers and operators.
\begin{enumerate}[label=(\alph*)]
\item 
Not every student in CSE015 plays a musical instrument
\begin{itemize}
    \item 
    $\forall x(S(x) \rightarrow M(x))$
\end{itemize}
\item 
A person is either a student in CSE015 or plays a musical instrument, but not both.
\begin{itemize}
    \item 
    $\exists x(S(x) \oplus M(x))$
\end{itemize}
\item 
There exists at least one student in CSE015 who does not play a musical instrument.
\begin{itemize}
    \item 
    $\exists x(S(x) \rightarrow \lnot(M(x))$
\end{itemize}
\end{enumerate}
\item 
\textbf{Question 3:} Logical Equivalence 
\begin{equation}
    \forall x(A(x) \wedge B(x)) \equiv \forall x(A(x) \rightarrow B(x))
\end{equation}
\begin{itemize}
    \item 
    This logical equivalence is not valid. If $ A(x)$ is false and $B(x)$ is true, then the left side of the equation would be false. However, the right side of the equation would be true because $A(x) \rightarrow (B(x)$ is true when $B(x)$ is true, regardless of $A(x)$ being false. Therefore, since $\forall x$ is stating that these predicates are equivalent through all the domain, this logical equivalency is not valid because they are not equivalent through all domain. The same can be noted when both $A(x)$ and $B(x)$ are false. The left is false, and the right is true.
    \item 
    Truth table: \\
    \begin{tabular}{|c|c|c|c|c|} \hline
    $A(x)$ & $B(x)$ & $A(x) \wedge B(x)$ & $A(x) \rightarrow B(x)$ & $A(x) \wedge B(x) \equiv A(x) \rightarrow B(x)$ \\
    \hline
    T&T&T&T&T \\
    \hline
    T&F&F&F&T \\
    \hline
    F&T&F&T&F \\
    \hline
    F&F&F&T&F \\
    \hline
    \end{tabular}
\end{itemize}
\item 
\textbf{Question 4:} Nested Quantifiers \\
The following two statements are defined for real numbers: \\
$A(x,y)$ is the statement $xy = 0$. \\
$B(x,y)$ is the statement $x + y = 0$ \\
For each of the following formulas determine their truth values. 
\begin{enumerate}[label=(\alph*)]
\item 
$\exists x \forall y A(x,y)$ 
\begin{itemize}
    \item 
    True. There is one $x$ for all $y$ which validates $xy = 0$, that is $x = 0$. 
\end{itemize}
\item
$\exists x \exists y B(x,y)$ 
\begin{itemize}
    \item 
    True. There is one $x$ and one $y$ which validates $x + y = 0$. $x = -1$ and $y = 1$ or vice-versa. Any positive integer with its counterpart negative will result in this.
\end{itemize}
\item 
$\forall x \exists y A(x,y)$
\begin{itemize}
    \item 
    True. For all $x$ there is one $y$ which validates $xy = 0$, that is $y = 0$
\end{itemize}
\item 
$\exists x \forall y (A(x,y) \wedge (B(x,y))$
\begin{itemize}
    \item 
    False. There is one $x$ for all $y$ which validates $A(x,y)$, $x = 0$, but not $B(x,y)$. Because the formula states that both functions are valid with $\exists x \forall y$, this is false. 
\end{itemize}
\item 
$\exists x \exists y (A(x,y) \wedge \lnot B(x,y))$
\begin{itemize}
    \item 
    True. There is at least one combination of $x$ and $y$ which validates $A(x,y)$ and does not validate $B(x,y)$. That combination is any integer $x$ or $y$ paired with 0 for either $x$ or $y$.
\end{itemize}
\end{enumerate}
\item 
\textbf{Question 5:} Negating formulas with Nested Quantifiers \\
Write the negation of the following statements so that the negation never appearse in front of a quantifier or of an expression involving logical connectives.
\begin{enumerate}[label=(\alph*)]
\item 
$\exists x \exists y (P(x) \rightarrow Q(y))$
\begin{itemize}
    \item 
    $ \lnot \exists x \exists y ( P(x) \rightarrow Q(y))$ becomes: 
    \item 
    $\exists x \exists y (\lnot P(x) \rightarrow \lnot Q(y))$
\end{itemize}
\item 
$\exists y(\exists x A(x,y) \vee \forall xB(x,y))$
\begin{itemize}
    \item 
    $\lnot \exists y(\exists x A(x,y) \vee \forall x B(x,y))$ becomes: 
    \item
    $\exists y(\exists x \lnot A(x,y) \vee \forall x \lnot B(x,y))$
\end{itemize}
\end{enumerate}
\end{enumerate}


\end{document}
